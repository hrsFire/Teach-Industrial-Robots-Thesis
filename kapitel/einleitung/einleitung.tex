Industrieroboter erfreuen sich heutzutage großer Beliebtheit bei Unternehmen um Produkte schnell und effizient herstellen zu können. Die Einsatzbereiche reichen hierbei von Laserschneiden über Schweißen und Fräsen bis hin zu Montageaufgaben, welche mitunter nur schwer von Personen durchgeführt werden können oder sogar lebensbedrohlich für Menschen sind. Zur Vermeidung von lebensbedrohlichen Gefahren und zur Erleichterung von Vorgängen, können Industrieroboter den Menschen hierbei effizient unterstützen und schwere Aufgaben erleichtern. Nichtsdestotrotz müssen von den Industrieroboterherstellern die geltenden Maschinenrichtlinien eingehalten werden um die Arbeit mit Menschen so gefahrlos wie nur möglich zu gestalten. Ein Beispiel hierfür ist das Erlernen bzw. Teachen von neuen Posen für einen Industrieroboter, sodass dieser seine Aufgaben im späteren Verlauf autonom durchführen kann. Hierzu muss sich die bedienende Person mitunter sehr nahe beim Industrieroboter befinden um Kollisionen mit Objekten frühzeitig erkennen und vermeiden zu können, aber auch um hohe Genauigkeiten für die zu erlernenden Zielposen zu erreichen. Zur Vermeidung von lebensbedrohlichen Situationen müssen die Industrieroboter beim Teachen deswegen unter anderem die Geschwindigkeit und Beschleunigung der Gelenke des Industrieroboters begrenzen.

\section{Umfeld}
Zum Teachen von Industrierobotern werden heutzutage vermehrt Teach Pendants eingesetzt, da durch deren Einsatz von Joystick und 3D-Maus als Eingabemethode sehr feine Bewegungen realisiert werden können. Zudem bietet das Teach Pendant durch das ergonomische Bedienen des \quoteMark{Enable device}-Schalters eine für Menschen sichere Möglichkeit einen Industrieroboter sicher zu bedienen. Bei zu starker oder zu schwacher Betätigung des \quoteMark{Enable device}-Schalters wird vom Teach Pendant eine Gefahrensituation angenommen und der Industrieroboter zum Stillstand gebracht um die bedienende Person vor lebensbedrohlichen Situationen zu schützen. Im Großen und Ganzen kann gesagt werden, dass ein Teach Pendant ein Multifunktionsgerät zur Programmierung von Industrierobotern darstellt und ohne großen Rechenaufwand und entsprechend komplizierten Algorithmen zum selbstständigen Finden der Wege auskommt. Diese Designentscheidung war unter anderem notwendig, da der Rechenaufwand zum selbständigen Finden von Wegen zu der Zeit der ersten Teach Pendants viel zu hoch gewesen wäre. Zudem ist es nicht immer möglich die Wege selbstständig von einer Software finden zu lassen, da zumeist zuwenig Informationen über die Umgebung bereitsteht. Daher musste eine alternative Möglichkeit entwickelt werden um Industrierobotern Posen beibringen zu können, welche anschließend autonom abgefahrt werden können. Die Möglichkeit Posen mittels Teach Pendant dem Industrieroboter zu erlernen hat sich aufgrund des Rechenaufwands daher bis heute als sehr hilfreich erwiesen. Aus diesem Grund haben sich Teach Pendants als De-facto-Standard in der Programmierung von Industrierobotern entwickelt. Die Funktionalität der Teach Pendants wurde zudem um Hilfsfunktionen, wie z.B. Bewegungsmodi und Beschleunigungsanpassungen, und Hardwarefunktionalitäten, wie z.B. einen Touchpad, erweitert, um die Bedienung des Teach Pendants und so auch des Industrieroboters noch intutitiver zu gestalten.

\section{Problemstellung}
Obwohl Teach Pendants sehr weit verbreitet sind und sich über die Jahre als sehr hilfreich beim Teachen von Industrierobotern erwiesen haben, wird dennoch unter anderem durch den rasanten Anstieg der Rechenkapazität an neuen Eingabemethoden für Industrieroboter geforscht. Der am zukunftsweisendste Ansatz liegt in der Verwendung von NUI um der bedienenden Person die Bedienung des Industrieroboters mittels des eigenen Körpers so intutitiv wie nur möglich zu gestalten. Eine Möglichkeit eine natürliche Benutzterschnittstelle zu entwickeln besteht in der Verwendung von Gesten. Gesten sind intutitiv zu erlernen und einfach anzuwenden, wodurch die Verwendung eines zusätzlichen Geräts, wie z.B. eines Teach Pendants, entfällt. Bei schweren Eingabegeräten, welche in der Hand gehalten werden, ist dies besonders spürbar, da ein längeres Bedienen des Geräts zu Ermüdungserscheinungen in den Händen und Armen führen kann. In der Gestenerkennung wird aus diesem Grund eine Möglichkeit gesehen, eine einfache und intutitive Bedienung mit zusätzlich noch ergonomischerer Bedienung als mit einem Teach Pendant zu ermöglichen. Die Sicherheitsaspekte dürfen dabei jedoch aber auch nicht vernachlässigt werden, da die Maschinenrichtlinien genauestens spezifizieren wie die zu bedienende Maschine mit Menschen interagieren muss um lebensbedrohliche Situationen zu vermeiden. Unter anderem muss hierfür die Geschwindigkeit und die Beschleunigung des Industrieroboters gedrosselt werden um unbeabsichtigte Bewegungen des Industrieroboters zeitgerecht ausweichen zu können.

\section{Zielsetzung} \label{sec:zielsetzung} %Motivation
In dieser Arbeit stellt daher das Erstellen, Testen und Analysieren einer Gestenerkennung, welche die Tiefeninformationen von einer Tiefenkamera erhält, im Vordergrund. Hierbei sollen ergonomische Gesten entwickelt werden, welche nur schwer unbeabsichtigt durchgeführt werden können, die Genauigkeit der Gestenerkennung und der Zielposen analysiert und Latenztests durchgeführt werden. Zur genauen Umsetzung soll eine Azure Kinect als Tiefenkamera eingesetzt und der \quoteMark{WidowX 200}-Lernroboter zur Umsetzung und Überprüfung der Gestenerkkenung verwendet werden. Die zu erstellende Gestenerkennung soll jedoch nicht nur auf die Azure Kinect und den \quoteMark{WidowX 200}-Lernroboter beschränkt sein. Hierfür sollen deshalb Schnittstellen geschaffen werden, welche das einfache Austauschen der Tiefenkamera- und Roboter-Komponente ermöglichen sollen. Aus diesem Grund soll zudem ermittelt werden ob eine Anbindung über ROS hinsichtlich der Latenzen eine ernst zu nehmende Alternative zu einer direkten Kommunikation mit einem Industrieroboter darstellt. Daher soll eine konkrete Kommunikation über ROS und eine direkte Kommunikation mit dem \quoteMark{WidowX 200}-Lernroboter implementiert werden. Zu guter Letzt sollen die beiden Implemntierungen auf Grundlage von Latenztests verglichen werden. Im Anschluss kann ein Fazit gezogen werden ob ROS den niedrigen Latenzenansprüchen von Industrierobotern gerecht werden kann.
