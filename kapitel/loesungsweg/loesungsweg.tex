\textcolor{red}{TODO:\\
Verwendeter Industrieroboter (Lernroboter, WidowX 200, warum?)\\
Was wird in diesem Kapitel beschrieben?
}

%-----------------------------------------------

% LXC: https://serverfault.com/questions/630220/how-do-i-configure-lxc-to-allow-the-use-of-sched-rr-in-a-container
% Scheduler wechseln:
% cat /sys/block/sda/queue/scheduler
% https://www.thomas-krenn.com/de/wiki/Linux_I/O_Scheduler#Deadline


%--------------
%sudo apt install schedtool # http://manpages.ubuntu.com/manpages/xenial/man8/schedtool.8.html
%
%"A privileged user can change the priority policy of a process with the schedtool program[7]:ln 326, 373 or it is done by a program itself.[7]:ln 336 The priority class can be manipulated at the code level with a syscall like sched_setscheduler only available to root,[11] which schedtool uses.[12]"
%# https://en.wikipedia.org/wiki/Brain_Fuck_Scheduler
%sudo chrt -p $(pidof -s bash)


% "the default scheduler is CFS, the "Completely Fair Scheduler"
% http://manpages.ubuntu.com/manpages/focal/en/man7/sched.7.html
% http://manpages.ubuntu.com/manpages/eoan/en/man7/sched.7.html
% https://en.wikipedia.org/wiki/Completely_Fair_Scheduler
% https://www.kernel.org/doc/html/latest/scheduler/sched-design-CFS.html
%-------------

% https://www.researchgate.net/publication/331290349_The_real-time_linux_kernel_A_survey_on_Preempt_RT

% -----------


% sudo chrt -d --sched-runtime 1000000 --sched-deadline 5000000 --sched-period 5000000 -p  0 57802
% https://access.redhat.com/solutions/3742421
% http://manpages.ubuntu.com/manpages/cosmic/de/man1/chrt.1.html
% https://lwn.net/Articles/743740/
% # nanoseconds for the parameters

% TODO: Wie lange ist für uns Echtzeit? -> Messungen!

%-----------------------------------------------


% Dead Man's Switch: irgendetwas in der Hand halten?
% Wenn sich keine Person vor dem Tiefensesor befindet, dann werden auch keine Bewegungen ausgeführt.

% Azure Kinect: verwendete Komponente: Bodytracking SDK
% verwendeter Tiefensensor-Modus -> warum?
% Entwicklung ohne Azure-Cloud

% Intel Real Sense Camera ZR300

\section{Allgemeine Anforderungen}
% weiche Echtzeitfähigkeit
% Durchsatz, Latenzen, ...

\subsection{Funktionale Anforderungen}


\subsection{Nichtfunktionale Anforderungen}


\section{Einrichtung des Testsystems}
\textcolor{red}{TODO:\\
Linux-Container\\
ROS\\
\\
Was ist ein Linux-Container? Vorteile?\\
% https://www.techdivision.com/blog/lxc-vs-docker-wir-setzen-bei-techdivision-inzwischen-verstaerkt-auf-lxc.html
% https://www.webhod.de/lxc-und-lxd-was-sind-linux-container/
% https://www.linux-magazin.de/ausgaben/2015/05/lxd/
\\
Anhang: Installationsanleitung
}


\subsection{Simulationsumgebungen}
\textcolor{red}{TODO:\\
Gazebo, vRep %, Coppelia Sim, ABB, ...
https://www.ros.org/integration/
}


\section{Anbindung an ROS}
\textcolor{red}{TODO:\\
Anbindung an ROS
}


\section{Einbinden von Nicht-ROS-Komponenten}
\textcolor{red}{TODO:\\
WidowX 200 (direkte Verbindung) \& Azure Kinect SDKs\\
Probleme und Besonderheiten
}


\subsection{ROS-Packages}
\textcolor{red}{TODO:\\
Verwendete ROS Packages
}


\section{Gesten}


\subsection{Arten von Gesten}
\textcolor{red}{TODO:\\
Welche mögliche Gesten gibt es?
}

\subsection{Auswahl anhand der Ergonomie}
\textcolor{red}{TODO:\\
Welche Gesten sind zu empfehlen, welche sollten eher nicht gewählt werden? Gibt es vielleicht Bilder?\\
Begründung der Implementierten Gesten für die jeweiligen Aktionen (Joint Mode, ...) und warum diese sinnvoll sind
}


\section{Aufbau des Gesten-Roboter-Frameworks}
\textcolor{red}{TODO:\\
Gesten-Roboter-Framework mit UML erklären\\
Wie kann es gestartet werden?\\
Was sind die Vorraussetzungen um das Backend zu starten?\\
ROS-Dependency Tree\\
Klassenhierarchie und Vererbungsmöglichkeiten (Erweiterbar für jede Arte von Roboter \& Tiefenkamera, welche mindestens die folgenden Anforderungen erfüllt ..., ...)
}


\section{Messvoraussetzungen}
\textcolor{red}{TODO:\\
Durchsatz, Latenzen, ...\\
Informationen aus Datenblätter\\
Zeit zur Erkennung von Gesten (Bodytracking SDK: beinhaltet Latenz von Aussenden des Infrarot-Lichtimpuls bis zum Empfang zur Absorption des Lichtimpuls, durch Belichtungszeit bis hin zur Latenz über das \quoteMark{USB 2.0}-Kabel)?\\
Mit Simulationsumgebung durchführen
}
