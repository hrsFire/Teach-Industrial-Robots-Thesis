%Methodenwahl % Begründung der Auswahl

Industrieroboter können eine Vielzahl von verschiedenen Aufgabengebieten übernehmen. Die Aufgabengebiete reichen hierbei von Bedinungs- über Bearbeitungs- bis hin zu Montageabläufen \textcolor{red}{Literatur}. Die Schwierigkeit besteht hierbei zunächst die Anforderungen an den erforderlichen Industrieroboter zu definieren und wenn möglich einzugrenzen. Da die Kommunikation mit und ohne ROS getestet werden soll ist es von Vorteil einen Industrieroboter mit einer bereits vorhandenen ROS-Unterstützung zu verwenden. Zudem ist es erforderlich eine direkten Schnittstelle auf den Industrieroboter zu besitzten um die gewünschten Abläufe mit einer direkten Kommunikation testen und einen Vergleich auf Grund des Zeitverhaltens anstellen zu können. Die Interoperabilität mit Simulationsumgebungen ist auch abzuwägen, da die Tests mithilfe einer Simulationsumgebung ohne Sicherheitsbedenken für die bedienende Person durchgeführt werden können und zudem für Regressionstests der Gestenerkennungssoftware praktikabel ohne einen realen Roboter durchgeführt werden können. Im weiteren Verlauf muss zudem die Entscheidung für einen Tiefensensor gemacht werden um eine Gestenerkennung, welche zur Steuerung des Inustrieroboters eingesetzt werden soll, realiseren zu können. Bei der Gestenerkennung steht vor allem die Ergonomie und die Sicherheit der bedienenden Person im Vordergrund. Hierbei muss unter anderem beachtet werden, dass zufällige Gesten nicht als Aktion gewertet werden, da dies ansonsten die Sicherheit der bedienenden Person negativ beinflussen kann.


\section{Industrieroboter}
% Arten & Vergleiche
% Genauigkeit
% Wie schnell können diese reagieren? (Zeitverhalten) je nach Roboterarm unterschiedlich
% DOF
% verschiedene Teile
% besteht im Allgemeinen aus dem Manipulator (Roboterarm), der Steuerung und einem Effektor (Werkzeug, Greifer etc.)


\section{Arten von Teach Pendants}
% Arten von Handprogrammiergeräten
% Aufzählen
% Arten & Vergleiche
% Genauigkeit
% Wie funktioniert ein Teach Pendant
% Notaus stopp


\section{Arten von Zeitverhalten}
% Echtzeitverhalten
%    weiches & hartes Echtzeitverhalten
% Normales Zeitverhalten


\section{Sicherheitsanforderungen im Umgang mit Roboterarmen}


\section{ROS}
% Aufbauend auf dem Vergleich der Entwicklungsplattformen werden nun die für dieses Projekt am Besten geeigneten Werkzeuge vorgestellt.
% ROS 1 (LTS) vs ROS 2


\subsection{Architektur}


\subsection{Packages}


\subsection{Zukünftige Entwicklung}
% Community driven, ROS Foundation, ...


\subsection{Anbindungsmöglichkeiten/Interoperabilität}


\section{Simulations- und Testumgebungen}
% Gazebo, vRep, Coppelia Sim, ABB, ...


\section{Tiefensensoren}
% Arten und Vergleich (Vor- und Nachteile)
% Azure Kinect, ...
% Genauigkeit, Techniken, ...


\section{Gesten}
% Mögliche Gesten, Welche Gesten nicht


\section{Ergonomie}
