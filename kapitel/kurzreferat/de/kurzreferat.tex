Das Teach Pendant wird bereits seit seiner Ersterscheinung dazu genutzt um Industrierobotern Zielposen beizubringen, damit diese die Zielposen daraufhin autonom anfahren können. Durch den auf dem Teach Pendant zur Eingabe verbauten Joystick oder die 3D-Maus ist es damit möglich sehr feine Bewegungen für den Industrieroboter zu realisieren. Außerdem bietet das Teach Pendant die Möglichkeit mit geringem Rechenaufwand einem Industrieroboter Zielposen beizubringen. Diese Designentscheidung war unter anderem notwendig, da der Rechenaufwand zum selbständigen Finden von Wegen zu der Zeit der ersten Teach Pendants viel zu hoch gewesen wäre. Zudem ist es nicht immer möglich die Wege selbstständig von einer Software finden zu lassen, da zumeist zuwenig Informationen über die Umgebung bereitsteht. Eingaben mittels natürlichen Benutzerschnittstellen, wie z.B. Gesten, waren zudem zu dieser Zeit noch nicht ausreichend präzise genug realisierbar.\\

Durch den rasanten Anstieg der Rechenkapazität und den stetigen Erfolgen bei der Erforschung von künstlichen neuronalen Netzwerken sind heutzutage Gestenerkennungssysteme in das Blickfeld der Forschung gelangt. Gestenerkennungssysteme versprechen intuitive und leicht zu erlernende Bedienungskonzepte. Durch den Einsatz von Gesten kann zudem auf ein zusätzliches Gerät in den Händen verzichtet werden, wodurch das Nutzererlebnis gesteigert werden kann. Dies kann bei schweren Eingabgeräten nicht nur die Arme sondern auch die Hände vor Ermüdungserscheinungen schonen.\\

Das Ziel dieser Arbeit soll es daher darstellen, ein Gestenerkennungssystem mit und ohne ROS-Anbindung zu erstellen, testen und analysieren. Die Gesteninformationen werden von einer Azure Kinect bereitgestellt, wobei jedoch die Tiefenkamera- und Roboter-Komponente austauschbar bleiben soll. Als Industrieroboter wird hierbei der \quoteMark{WidowX 200}-Lernroboter eingesetzt, da dieser aufgrund seiner kleinen Bauform einfach und effizient zum Testen von neuen Funktionalitäten eingesetzt werden kann. Die zu entwickelnden Gesten sollen vor allem hohe Ergonomie bereitstellen und vor unbeabsichtigter Durchführung gesichert sein. Da die Gesten zuvor entwickelt werden müssen, ist es daher notwendig die Gesten praktischen Tests zu unterziehen um die Zuverlässigkeit der Gesten bewerten zu können. Bei den Tests spielt auch die Genauigkeit der Gestenerkennung eine große Rolle um die Zuverlässigkeit der Gesten evaluieren zu können. Die Genauigkeit der erreichten Zielposen des \quoteMark{WidowX 200}-Lernroboters müssen analysiert und Latenztests durchgeführt werden um das Gestensystem als Gesamtes zu bewerten.
