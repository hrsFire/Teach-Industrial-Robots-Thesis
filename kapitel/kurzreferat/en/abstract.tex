The Teach Pendant has been used since its first appearance to teach industrial robots target poses so that they can then approach the target poses autonomously. The joystick or 3D mouse installed on the teach pendant for input makes it possible to realize very fine movements for the industrial robot. In addition, the teach pendant offers the possibility of teaching target poses to an industrial robot with little computing effort. This design decision was necessary, among other things, because the computational effort required to find paths independently would have been far too high at the time of the first teach pendants. In addition, it is not always possible to have the paths found independently by a software, since there is usually too little information available about the environment. Furthermore, inputs via natural user interfaces, such as gestures, were not yet sufficiently precise enough at that time.\\

Due to the rapid increase in computing capacity and the constant success in the research of artificial neural networks, gesture recognition systems have become the focus of research today. Gesture recognition systems promise intuitive and easy to learn operating concepts. The use of gestures also makes it possible to do without an additional device in the hands, thus enhancing the user experience. In the case of heavy input devices, this can protect not only the arms but also the hands from signs of fatigue.\\

The goal of this thesis is therefore to create, test and analyze a gesture recognition system with and without ROS connection. The gesture information is provided by an Azure Kinect, but the depth camera and robot components should remain interchangeable. The \quoteMark{WidowX 200} learning robot will be used as the industrial robot for this purpose, as its small size means that it can be used simply and efficiently to test new functionalities. The gestures to be developed should above all provide high ergonomics and be protected against unintentional execution. Since the gestures have to be developed beforehand, it is therefore necessary to subject the gestures to practical tests in order to evaluate their reliability. The accuracy of the gesture recognition also plays an important role in the tests to evaluate the reliability of the gestures. The accuracy of the achieved target poses of the \quoteMark{WidowX 200} learning robot must be analyzed and latency tests must be performed to evaluate the gesture system as a whole.
