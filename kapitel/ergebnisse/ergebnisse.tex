% da ein Delay von < 50 ms aufgrund der Gestenerkennung nicht eingehalten werden kann muss die Berechnung des neuronalen Netzwerks der Gestenerkennung noch schneller werden bzw. auf speziell darauf ausgelegter Hardware (FPGA) ausgeführt werden

% ------------

% Azure Kinect: Belichtungszeit: 1,6 ms bis zu 20,3 ms (https://docs.microsoft.com/en-us/azure/kinect-dk/hardware-specification) -> TODO: Passive IR einschalten?
% Realsens: Belichtungszeit: 900 ns (https://www.imveurope.com/press-releases/intel-realsense-lidar-depth-camera-l515) (https://softei.com/framos-claims-hi-res-intel-realsense-lidar-depth-camera-is-worlds-smallest/) (https://venturebeat.com/2019/12/11/intels-new-realsense-camera-packs-a-lidar-sensor-for-enhanced-depth-perception/) -> Distanz?
% https://www.intelrealsense.com/compare-depth-cameras/
% https://www.intelrealsense.com/depth-camera-d435/

% https://docs.microsoft.com/en-us/windows/mixed-reality/ISSCC-2018



%----------------

% https://feedback.azure.com/forums/920053-azure-kinect-dk/suggestions/38129473-body-tracking-without-cudnn
% https://feedback.azure.com/forums/920053-azure-kinect-dk/suggestions/39945454-legacy-body-tracking-like-kinect-v2
% https://github.com/microsoft/Azure-Kinect-Sensor-SDK/issues/1080



%-----------\\
% konkrete Aussage zur Performanz, Stabilität, Modularität, Hohe Skalierbarkeit,
% Erweiterbarkeit

% Erfahrungen

% Mögliche Prognosen

% Limitierungen

% Entwicklungswerkzeuge


% --------------\\

% Andere Vergleiche in der Literatur


\textcolor{red}{TODO:\\
Was wird in diesem Kapitel beschrieben?\\
Visualisierung:\\
* Diagramme\\
* Mittelwert, Standardabweichung, Varianz, ...
}

% weiche Echtzeitfähigkeit
% Durchsatz, Latenzen, ...

% Die Gestensteuerung des WidowX 200, welches auf dem für diese Arbeit erstellten Gesten-Roboter-Framework basiert und als Basis für andere gestengesteuerte Roboter dienen kann, wird auf die Ergonomie, Echtzeitfähigkeit, Genauigkeit der Gestenerkennung und Genauigkeit der Zielpositionen hin überprüft. Zudem wird ROS 1 auf die Netzwerklatenz hin überprüft um darauf basierend eine Einschätzung zu gegeben ob sich ROS 1 für echtzeitfähige Systeme eignet.

\section{Latenz der Gestenerkennung}
\textcolor{red}{Azure Kinect im CPU-Modus anstatt mit Grafikkartenbeschleunigung, weil keine Nvidia Grafikkarte vorhanden war und der kleinste gemeinsame Nenner gewählt wird}

\section{Latenz der Übertragung mit und ohne ROS-Anbindung}
\textcolor{red}{TODO:\\
WidowX 200\\
Verschiedene Netzwerkkonstellationen simulieren? % aus dem Sicherheitsaspekt sollte für ROS 1 ein eigens für den Roboter abgeschottenes Netzwerk verwendet -> dadurch meisten weniger Latenzen, weil weniger über ROS übertragen wird?
% http://wiki.ros.org/Topics
% Topic statistics
}

\section{Genauigkeit der Ziele}
\textcolor{red}{TODO:\\
Im Vergleich zu einer anderen Eingabemethode\\
PS3-Controller: https://www.trossenrobotics.com/widowx-200-robot-arm.aspx
}

\section{Ergonomie \& User Experience}
\textcolor{red}{TODO:\\
Wie gut ist das System einsetzbar? UX\\
Bewertung der Ergonomie über längere Zeitdauer (sehr subjektiv, jedoch aber versuchen die Ergebnisse auf die Allgemeinheit zu bezienen)\\
Während dem Teachen bewegt man sich oft um zu sehen ob der Endeffektor an der richtigen Stelle ist. Mit einer fest angebrachten Azure Kinect ist es daher schwer den eigenen Blickwinkel zu ändern\\
reflexartige und unbeabsichtigte Bewegungen
}
