\section{Zusammenfassung}

% Azure Kinect: 
%    Stärken der Kinect: Keine Notwendigkeit einer Kalibrierpose, Posenberechnung ohne getragene Hardware
% Problem: immer in Richtung der Azure Kinect schauen -> diese ist statisch

\section{Ausblick}

% ROS ist nicht das Allheilmittel sonder eine mögliche Implementierung die sich erst noch im industriellen Einsatz behaupten muss über die Jahre \cite{noauthor_why_dont_we_use_ros_nodate}

% Was könnte in Zukunft besser gemacht werden?

% Gibt es bessere Alternativen?

% Gibt es weiterführende Themen, welche behandelt werden könnten?


\textcolor{red}{
Weiteres Forschungspotenzial im Bereich der Sicherheit und Ergonomie würde vermutlich zudem in smarten Fußeinlagesohlen bestehen, da hierdurch die Hände zur Gestensteuerung frei bleiben würden. Zudem wäre es womöglich machbar sehr schnelle und reflexartige Druckverteilungen und Bewegungen zu Erkennen und den Industrieroboter bei dieser Art von Bewegung zu stoppen, wenn zu wenig oder kein Druck auf den Sohlen gemessen würde. Dieser Umstand könnte mehrere Gründe haben, da z.B. die Sohlen abgezogen wurden oder die bedienende Person ungewollt auf dem Boden liegt. Das Starten der Gestenerkennung könnte womöglich auch durch eine korrekt ausgeführte Kombination aus Fußbewegungen realisiert werden. --> TODO: Quelle; Vergleich zu Plattform fürs Teachen\newline
\newline
Eine andere Möglichkeit würde darin bestehen einen smarten Handschuh zu verwenden, welcher zu schnelle Bewegungen erkennt und die Bewegung des Industrieroboters anhält. -> TODO: Quelle
}
